% Kapitel 'Mathematische Grundlagen'
\chapter{Mathematische Grundlagen}
In den folgenden Abschnitten m�chte ich die mathematischen Grundprinzipien behandeln, auf denen nahezu alle Konzepte und Operationen der 3D-Grafik aufbauen.

% Kapitel �ber Vektoren
% Wozu werden welche Vektoren verwendet? Erl�uterung homogene Koordinaten. Schreibweisen von Vektoren.
\section{Vektoren}
Vektoren sind wohl die wichtigste Struktur in der 3D-Grafik. Sie werden nicht nur in ihrer dreidimensionalen Form eingesetzt, sondern auch in ihrer zweidimensionalen Form und um eine vierte Koordinate erweitert als Vektoren mit homogenen Koordinaten (dazu sp�ter mehr).

Trotz der universellen Verwendung gehen die ben�tigten Operationen nicht �ber den Schulstoff hinaus. Insbesondere das Skalarprodukt und das Kreuzprodukt werden sehr h�ufig ben�tigt, zum Beispiel um �ber den Zusammenhang
% Wie 'flacher' setzen?
$\vec{a}\cdot\vec{b} = \left|\vec{a}\right|\left|\vec{b}\right|\cos\alpha$
den von zwei Vektoren eingeschlossenen Winkel zu berechnen oder �ber das Kreuzprodukt eine Normale eines Polygons zu bestimmen.

Vektoren k�nnen dabei auf zwei Arten notiert werden: in Spaltenform


\subsection{Homogene Koordinaten}

Vierdimensionale Vektoren, also Vektoren mit \emph{homogenen Koordinaten}, dienen haupts�chlich dazu, mithilfe von Matrizen nicht nur lineare, sondern alle affinen Transformationen darstellen zu k�nnen.
% Beispiel aus http://de.wikipedia.org/wiki/Homogene_Koordinaten einf�gen.
Dazu wird eine vierte Koordinate $w$ eingef�hrt.
% Hyperebene: Eberly (3D game engine design), S. 9 (2.1.4 Homogeneous Transformations)
% Verwendung von w in Robotik-Matrizen?


\section{Matrizen}
Generelle Einleitung. Auf den ersten Blick sehen Matrizen genauso aus wie einfache Tabellen von Zahlen. Sie unterscheiden sich von ihnen jedoch dadurch, dass man mit ihnen rechnen kann, sie werden der linearen Algebra zugeordnet.

\subsection{Multiplikation}
Wie man zwei Matrizen multipliziert.

\subsection{Transposition}
Erkl�rung der Transposition einer Matrix.

\subsection{Determinante und Inverse}
Definition der Inversen einer Matrix. Um sie berechnen zu k�nnen, nimmt man die determinante zur Hilfe.

\section{Quaternionen}
Erkl�rung zu Quaternionen. Nicht ganz allt�gliches Konzept.

\subsection{Rechenregeln}
Kurze Zusammenfassung der Rechenregeln f�r Quaternionen.