% Kapitel 'Mathematische Grundlagen'
\chapter{Mathematische Grundlagen}
In den folgenden Abschnitten möchte ich die mathematischen Grundprinzipien behandeln, auf denen nahezu alle Konzepte und Operationen der 3D-Grafik aufbauen.

% Kapitel über Vektoren
% Wozu werden welche Vektoren verwendet? Erläuterung homogene Koordinaten. Schreibweisen von Vektoren.
\section{Vektoren}
Vektoren sind wohl die wichtigste Struktur in der 3D-Grafik. Sie werden nicht nur in ihrer dreidimensionalen Form eingesetzt, sondern auch in ihrer zweidimensionalen Ausprägung und um eine vierte Koordinate erweitert als Vektoren mit homogenen Koordinaten (dazu später mehr).

Trotz der universellen Verwendung gehen die benötigten Operationen nicht über den Schulstoff hinaus. Insbesondere das Skalarprodukt und das Kreuzprodukt werden sehr häufig benötigt, zum Beispiel um über den Zusammenhang
% Wie 'flacher' setzen?
$\vec{a}\cdot\vec{b} = \left|\vec{a}\right|\left|\vec{b}\right|\cos\alpha$
den von zwei Vektoren eingeschlossenen Winkel zu berechnen oder über das Kreuzprodukt eine Normale eines Polygons zu bestimmen.

Vektoren können dabei auf zwei Arten notiert werden, in Spaltenform
\begin{equation*}
 \vec{v} = \begin{pmatrix} x \\ y \\ z \end{pmatrix},
\end{equation*}
oder in Zeilenform
\begin{equation*}
 \vec{v} = \begin{pmatrix} x & y & z \end{pmatrix}.
\end{equation*}

Scheint der Unterschied zunächst noch rein kosmetischer Natur zu sein, zeigt sich spätestens beim Arbeiten mit Matrizen, dass die Entscheidung für eine der Formen doch erhebliche Konsequenzen nach sich zieht (siehe Kapitel \ref{transposition}). In der Grafikprogrammierung werden beide Varianten gleichermaßen verwendet; in diesem Dokument werde ich Spaltenvektoren verwenden, um den Bezug zur Schulmathematik zu wahren.

Eingebettet in Fließtext sind Zeilenvektoren zwar wesentlich platzsparender, man kann einen Spaltenvektor aber als transponierten Zeilenvektor $\vec{v} = \begin{pmatrix} x & y & z \end{pmatrix}^T$ anschreiben, um diesen Vorteil zu übernehmen (siehe Abschnitt \ref{transposition}).


\subsection{Homogene Koordinaten}
Homogene Koordinaten sind Koordinaten, die um eine zusätzliche Dimension -- meistens wird sie mit $w$ bezeichnet -- erweitert wurden. Das Pendant mit homogenen Koordinaten zu einem dreidimensionalen Vektor ist also ein Vektor im $\mathbb{R}^4$.

Homogene Koordinaten dienen dazu, alle affinen Transformationen eines Vektors durch eine Linearkombination seiner Koordinaten darstellen zu können. Ansonsten wäre man auf lineare Transformationen beschränkt, wenn eine Transformation über gleiche Koeffizienten der Linearkombination auf mehrere Vektoren angewendet werden soll (siehe Abschnitt \ref{homogenousmultiplication}).

Um einen \enquote{normalen Vektor} in einen Vektor mit homogenen Koordinaten zu überführen, setzt man einfach $w = 1$:
\begin{equation}
 \begin{pmatrix} x & y & z \end{pmatrix}^T \rightarrow \begin{pmatrix} x & y & z & 1 \end{pmatrix}^T
\end{equation}

Für den umgekehrten Fall dividiert man zuerst alle Komponenten durch $w$, um den Vektor auf $w = 1$ zu bringen, und lässt $w$ dann einfach weg:
\begin{equation}
 \begin{pmatrix} x & y & z & w \end{pmatrix}^T \rightarrow \begin{pmatrix} \frac{x}{w} & \frac{y}{w} & \frac{z}{w} \end{pmatrix}^T
\end{equation}
% Beispiele nach http://de.wikipedia.org/wiki/Homogene_Koordinaten

In der Grafikprogrammierung wird das meist so gelöst, dass von vornherein mit homogenen Koordinaten gearbeitet wird, die $w$-Komponente aber bei den meisten Berechnungen schlichtweg ignoriert wird. Sie wird nur bei Operationen, bei denen sie sich sinnvoll ändern könnte (wie bei der Multiplikation von Matrizen mit Vektoren), berücksichtigt. Auch wenn diese Vorgehensweise aus mathematischer Sicht nicht konsequent ist, stellt sie in der Praxis einen sinnvollen Kompromiss zwischen Performance\footnote{Performance: Leistung bzw. Geschwindigkeit eines Computerprogramms} und Flexibilität dar.

% Hyperebene: Eberly (3D game engine design), S. 9 (2.1.4 Homogeneous Transformations)
% Verwendung von w in Robotik-Matrizen?


\section{Matrizen}
Auf den ersten Blick gleichen Matrizen einfachen Tabellen von Zahlen, es gibt aber einen wesentlichen Unterschied: Für sie sind Rechenoperationen wie Addition und Multiplikation definiert.

Eine Matrix
\begin{equation}
 A = \begin{pmatrix}
   a_{11} & a_{12} & \cdots & a_{1m}\\
   a_{21} & a_{22} & \cdots & a_{2m}\\
   \vdots & \vdots & \ddots & \vdots\\
   a_{n1} & a_{n2} & \cdots & a_{nm}
 \end{pmatrix}
 \in \mathbb{R}^{m \times n}
\end{equation}
besteht aus $m$ Zeilen und $n$ Spalten, es handelt sich um eine $m \times n$-Matrix (sprich: \emph{m kreuz n}).

Matrizen mit $m = n$ werden als \emph{quadratische Matrizen} bezeichnet und nehmen bei vielen Operationen eine Sonderstellung ein.
% Besondere Eigenschaften der quadratischen Matrizen? Einige Operationen sind nur für quadratische Matrizen definiert, ...

Als \emph{Hauptdiagonale} wird die gedachte Linie von der linken oberen Ecke bis zur rechten unteren Ecke einer Matrix bezeichnet -- die Elemente der Hauptdiagonale sind also $a_{11}, a_{22}, \ldots, a_{nn}$. Quadratische Matrizen, die außerhalb der Hauptdiagonale nur nur Elemente mit dem Wert 0 haben, werden als \emph{Diagonalmatrizen} bezeichnet. Sie lassen viele Vereinfachungen in bestimmten Berechnungen zu. Diese werden aber hauptsächlich im zweiten großen Einsatzgebiet der Matrizenrechnung gebraucht, dem Lösen von linearen Gleichungssystemen. An dieser Stelle soll daher nicht weiter darauf eingegangen werden.
% siehe http://de.wikipedia.org/wiki/Diagonalmatrix. Stimmt das?!

% Nur Matrizen aus R

\subsection{Addition}
Zwei Matrizen der gleichen Dimensionen $m \times n$ werden addiert, indem man jeweils die Einträge der beiden Matrizen addiert:
\begin{align}
 (A + B)_{ij} = a_{ij} + b_{ij}%\\
% \nonumber\text{mit }1 \leq i \leq m \text{ und } 1 \leq j \leq n.
\end{align}
% Quelle für Notation: http://en.wikipedia.org/wiki/Matrix_(mathematics)
mit $1 \leq i \leq m$ und $1 \leq j \leq n$.

Folglich ist das Ergebnis der Addition wiederum eine $m \times n$-Matrix.

Die Matrizenaddition ist assoziativ und kommutativ. Es ist leicht zu erkennen, dass die Matrizenaddition -- analog zur Null bei der Addition von Skalaren -- ein neutrales Element besitzt: eine Matrix deren Elemente alle 0 sind, kurz \emph{Nullmatrix} genannt.

Die Addition wird in der 3D-Grafik äußerst selten gebraucht, da sie im Gegensatz zur Multiplikation und zur Inversion keine geometrische Entsprechung hat.

\subsection{Multiplikation}
Das Produkt $C$ zweier Matrizen $A \in \mathbb{R}^{m,o}$ und $B \in \mathbb{R}^{o,n}$ ist als
\begin{equation}
 c_{ij} = \sum_{k=1}^m{a_{ik} \cdot b_{kj}}
\end{equation}
mit $1 \leq i \leq m$ und $1 \leq j \leq n$ definiert ($C$ ist also $\in \mathbb{R}^{m \times n}$).

Etwas anschaulicher formuliert erhält man das $i$-te Element der $j$-ten Spalte des Ergebnisses, indem man das Punktprodukt der $i$-ten Zeile der linken Matrix mit der $j$-Zeile der rechten Matrix bildet -- kurz \enquote{Zeile mal Spalte}.

Es müssen also die Spaltenanzahl der linken Matrix und die Zeilenanzahl der rechten Matrix gleich sein (oben durch $o$ ausgedrückt), damit eine Multiplikation möglich ist. Das entstehende Produkt hat dabei die Dimensionen $m \times n$, also die Zeilenanzahl der linken Matrix und die Spaltenanzahl der rechten Matrix.

Ein Beispiel zur Veranschaulichung:
\begin{equation}
\begin{split}
 \begin{pmatrix}
  0 & 1 & 2 \\
  3 & 4 & 5
 \end{pmatrix}
 \cdot
 \begin{pmatrix}
  6 & 7 \\
  8 & -9 \\
  -10 & 11
 \end{pmatrix}
 =
 \begin{pmatrix}
  0 \cdot 6 + 1 \cdot 8 + 2 \cdot (-10) & 0 \cdot 7 + 1 \cdot (-9) + 2 \cdot 11 \\
  3 \cdot 6 + 4 \cdot 8 + 5 \cdot (-10) & 3 \cdot 7 + 4 \cdot (-9) + 5 \cdot 11
 \end{pmatrix}\\
 =
 \begin{pmatrix}
  0 + 8 - 20 & 0 - 9 + 22 \\
  18 + 32 - 50 & 21 - 36 + 55
 \end{pmatrix}
 =
 \begin{pmatrix}
   -12 & 13 \\
   0 & 40
 \end{pmatrix}
\end{split}
\end{equation}

Für die Multiplikation von quadratischen Matrizen gibt es ein neutrales Element, für das
\begin{equation}
 A \cdot E = E \cdot A = A
\end{equation}
gilt. $E$ wird als \emph{Einheitsmatrix} bezeichnet und ist eine Diagonalmatrix, deren Elemente entlang der Hauptdiagonale alle den Wert 1 haben. Es gibt natürlich für alle verschiedenen Größen $n \times n$ jeweils eine eigene Einheitsmatrix $E_n$. Im Normalfall ist aber ohnehin ersichtlich, welche Dimensionen die verwendete Einheitsmatrix haben muss, deswegen wird in der Praxis meistens auf den Index verzichtet.

Fasst man einen $n$-dimensionalen Spaltenvektor als $n \times 1$-Matrix auf, so kann die Multiplikation einer Matrix mit einem Vektor als Spezialfall der Matrizenmultiplikation gesehen werden. Das Ergebnis ist wiederum ein $n$-dimensionaler Vektor. Die Operation wird auch als \emph{Transformation} des Vektors bezeichnet, da in der Matrix Transformationen \enquote{gespeichert} sein können, die durch die Multiplikation auf den Vektor angewendet werden (siehe Kapitel \ref{transformation}).

Eine Multiplikation mit der Einheitsmatrix verändert den Vektor naheliegenderweise nicht:

\begin{equation}
 \begin{pmatrix}
  1 & 0 & 0 & 0 \\
  0 & 1 & 0 & 0 \\
  0 & 0 & 1 & 0 \\
  0 & 0 & 0 & 1
 \end{pmatrix}
 \cdot
 \begin{pmatrix}
  2 \\
  3 \\
  4 \\
  1
 \end{pmatrix}
 =
 \begin{pmatrix}
  1 \cdot 2 + 0 \cdot 3 + 0 \cdot 4 + 0 \cdot 1 \\
  0 \cdot 2 + 1 \cdot 3 + 0 \cdot 4 + 0 \cdot 1 \\
  0 \cdot 2 + 0 \cdot 3 + 1 \cdot 4 + 0 \cdot 1 \\
  0 \cdot 2 + 0 \cdot 3 + 0 \cdot 4 + 1 \cdot 1
 \end{pmatrix}
 =
 \begin{pmatrix}
  2 \\
  3 \\
  4 \\
  1
 \end{pmatrix}
\end{equation}

\label{homogenousmultiplication}
Jetzt wird auch der Hintergrund der Verwendung von homogenen Koordinaten klarer: Wenn eine Matrix auf mehrere Vektoren angewendet werden soll, dann könnten ohne diese nur lineare Abbildungen dargestellt werden. Für die Darstellung aller \emph{affinen Transformationen} wäre ein zusätzlicher Schritt notwendig, beispielsweise die Addition eines Vektors: $\vec{x'} = A \cdot \vec{x} + \vec{b}$. Durch die Verwendung von homogenen Koordinaten können diese Schritte in einer Matrix zusammengefasst werden. Außerdem sind Berechnungen wie Projektionen möglich, die eine \emph{Division} aller Koordinaten durch eine der Koordinaten beziehungsweise durch eine Linearkombination der Koordinaten erfordern (siehe Kapitel \ref{projection}).


\subsection{Transposition}
\label{transposition}
Bei der Transposition werden die Zeilen und Spalten einer Matrix vertauscht, die Matrix wird quasi entlang ihrer Hauptdiagonale gespiegelt. Folglich wird eine $m \times n$-Matrix zu einer $n \times m$-Matrix. Wird eine $n \times 1$-Matrix, also ein Spaltenvektor, transponiert, ergibt sich daher eine $1 \times n$-Matrix, also der entsprechende Zeilenvektor.

Beispiel:
\begin{equation}
  \begin{pmatrix}
    1 & -2 & 3 \\
    4 & 5 & -6
  \end{pmatrix}^T
  =
  \begin{pmatrix}
    1 & 4 \\
    -2 & 5 \\
    3 & -6
  \end{pmatrix}
\end{equation}

Wird die Transposition zwei Mal auf eine Matrix angewendet, hebt sie sich naheliegenderweise auf, es gilt also
\begin{equation}
 (A^T)^T = A.
\end{equation}

Bezüglich Addition ist die Transposition distributiv, bezüglich Multiplikation distributiv unter Umkehrung der Reihenfolge, es gilt also
\begin{equation}
 (A + B)^T = A^T + B^T
\end{equation}
und
\begin{equation}
\label{transpositionmultiplication}
 (A \cdot B)^T = B^T \cdot A^T.
\end{equation}

Der Zusammenhang aus Gleichung \ref{transpositionmultiplication} zieht einen wesentlichen Unterschied in der Verwendung von Spalten- und Zeilenvektoren nach sich. Dazu ein kleines Beispiel: Gegeben sei ein Spaltenvektor $\vec s$ und eine Matrix $A$, durch Multiplikation aus mehreren Teilmatrizen $A_1, A_2, \dots, A_n$ kombiniert, die jeweils einer Transformation entsprechen. Der Vektor wird nun von \emph{rechts} an die Matrix multipliziert:
\begin{equation}
 \vec{s'} = A \cdot \vec{s} = A_1 \cdot A_2 \cdot \ldots \cdot A_n \cdot \vec{s} = A_1 \cdot A_2 \cdot \ldots \cdot \left( A_n \cdot \vec{s} \right).
\end{equation}
Wie durch die Klammernsetzung angedeutet, verhält sich der Ergebnisvektor $\vec{s'}$, als hätte man die Teiltransformationen der Reihe nach beginnend mit $A_n$, also der zuletzt zu $A$ multiplizierten Matrix, auf den Vektor angewendet.

Wird nun statt einem Spaltenvektor ein Zeilenvektor $\vec z = \vec{s}^T$ verwendet, ergibt sich aus Gleichung \ref{transpositionmultiplication} (wie auch aus den für die Multiplikation nötigen Dimensionen der Argumente), dass ein Zeilenvektor von \emph{links} an die \emph{transponierte} Matrix multipliziert werden muss. Wenn diese wieder aus $n$ Teiltransformationen zusammengesetzt wird, reicht es nicht, die einzelnen Matrizen zu transformieren, es muss zusätzlich noch die Multiplikationsreihenfolge der Teilmatrizen umgekehrt werden:
\begin{equation}
 \vec{s'}^T = \vec{s}^T \cdot A^T = \vec{s}^T \cdot A_n^T \cdot A_{n-1}^T \cdot \ldots \cdot A_1^T.
\end{equation}
Dies ist insofern erwähnenswert, als in der 3D-Grafik zwei große Standards existieren, von denen einer Spaltenvektoren, der andere Zeilenvektoren verwendet (mehr dazu in Kapitel  	\ref{direct3dopengl}) und manche Algorithmen daher nicht 1:1 von einem Standard auf den anderen übertragbar sind.
% Welche?

Übrigens entspricht die transponierte Matrix bei Matrizen über $\mathbb R$ (und um diese soll es hier ausschließlich gehen) der \emph{adjungierten Matrix}. Aus diesem Grund werden die beiden Begriffe in der Fachliteratur zur Grafikprogrammierung gelegentlich synonym verwendet. Diese ist allerdings nicht mit der \emph{Adjunkten} zu verwechseln, die bei der Berechnung der inversen Matrix Verwendung findet.
% Quelle: http://de.wikipedia.org/wiki/Matrix_(Mathematik)#Die_transponierte_Matrix

\subsection{Determinante}
Die Determinante ist eine spezielle Funktion, die einer quadratischen Matrix $A$ eine Zahl $\det A$ zuordnet. Das Ergebnis kann auf verschiedene Weisen interpretiert werden, die bekannteste Verwendung ist sicherlich zur Volumenberechnung in der Vektorrechnung (der Betrag der Determinante von reellen Vektoren entspricht dem Spatprodukt). Wenn die Matrix als Koeffizientenmatrix eines linearen Gleichungssystems aufgefasst wird, besitzt dieses genau dann eine eindeutige Lösung, wenn ihre Determinante ungleich 0 ist (solche Matrizen werden als \emph{reguläre Matrizen} bezeichnet).

In der Grafikprogrammierung ist die Determinante selbst praktisch bedeutungslos, allerdings wird sie für ein Verfahren zur Inversion von Matrizen benötigt (siehe Abschnitt \ref{inversion}).

Die Determinante einer $2 \times 2$-Matrix
\begin{equation*}
 A = \begin{pmatrix}
      a & b \\
      c & d
     \end{pmatrix}
\end{equation*}
lautet
\begin{equation}
 \det A = ad - cb.
\end{equation}

Mit Hilfe des Laplace'schen Entwicklungssatzes können Determinanten aller höheren Dimensionen durch $2 \times 2$-Determinanten ausgedrückt werden. Dazu wird das Konzept der \emph{Minoren} eingeführt: Der Minor $M_{ij}$ einer Matrix $A \in \mathbb{R}^{n \times n}$ ist die Determinante der $(n-1) \times (n-1)$-Untermatrix, die durch Streichen der $i$-ten Zeile und der $j$-ten Spalte von A entsteht. Das Produkt
\begin{equation}
\label{cofactor}
 \tilde a_{ij} = (-1)^{i+j} M_{ij}
\end{equation}
wird als \emph{Kofaktor} von A bezeichnet. Nach dem Laplace'schen Erweiterungssatz kann die Determinante einer Matrix als Summe der Produkte ihrer Elemente mit den zugehörigen Kofaktoren \enquote{nach} einer beliebigen Zeile oder Spalte entwickelt werden, es gilt also mit $1 \leq k \leq n$:
\begin{equation}
 \det A = a_{k1} \tilde a_{k1} + a_{k2} \tilde a_{k2} + \cdots + a_{kn} \tilde a_{kn} = a_{1k} \tilde a_{1k} + a_{2k} \tilde a_{2k} + \cdots + a_{nk} \tilde a_{nk}.
\end{equation}
% Literatur?!

Hieraus lässt sich schon eine Eigenheit im Bezug auf die Transposition erahnen: Die Determinante einer Matrix verändert sich, wenn man die Matrix transponiert. Es gilt also:
\begin{equation}
\label{transposeddeterminant}
 \det A = \det A^T
\end{equation}

Nach diesem Verfahren lässt sich eine komplette Formel für die Determinante ableiten, was für die Geschwindigkeit der Berechnung am Computer von Vorteil ist\footnote{Dies gilt nur für Determinanten relativ kleiner Dimension, wie sie in der Grafikprogrammierung verwendet werden. Zur Berechnung höherdimensionaler Determinanten kann ein anderer algorithmischer Ansatz sinnvoller sein, da die Komplexität dieses Verfahrens ziemlich schnell steigt.}. Für eine $3 \times 3$-Matrix $B$ erhält man beispielsweise nach dem Vereinfachen
\begin{equation}
 \det B = b_{11} b_{22} b_{33} + b_{12} b_{23} b_{31} + b_{13} b_{21} b_{32} - b_{13} b_{22} b_{31} - b_{12} b_{21} b_{33} - b_{11} b_{23} b_{32}.
\end{equation}
Für die Berechnung der Determinante gibt es noch zahlreiche weitere Verfahren, darunter auch solche, die für den Einsatz mit großen Matrizen optimiert sind. In der 3D-Programmierung sind sie aber mehr oder weniger bedeutungslos, da es hier auf schnelle Berechenbarkeit bei kleinen Dimensionen ankommt.

% http://de.wikipedia.org/wiki/Determinante_(Mathematik)

\subsection{Inverse Matrix}
\label{inversion}
Eine Matrix $A$ kann eine inverse Matrix $A^{-1}$ besitzen, für die
\begin{equation}
 A \cdot A^{-1} = A^{-1} \cdot A = E
\end{equation}
gilt. $A^{-1}$ wird auch kurz als Inverse bezeichnet. Wie oben schon angemerkt, existiert nur zu reguläre Matrizen eine inverse Matrix, also zu quadratischen Matrizen, deren Determinante ungleich 0 ist. Nicht-quadratische Matrizen können prinzipiell keine inverse Matrix besitzen.
% allerdings können unter Umständen ... http://en.wikipedia.org/wiki/Moore-Penrose_pseudoinverse

Wenn man das Multiplizieren einer Matrix mit einem Vektor als Transformation betrachtet, dann entspricht die Multiplikation mit der inversen Matrix dem \emph{Rückgängig machen} der Transformation. Hat man also beispielsweise einen Vektor $\vec v$ mit der Matrix $A$ transformiert, erhält man nach einer Multiplikation mit der Inversen $A^{-1}$ wieder den ursprünglichen Vektor. Dies lässt sich auch direkt aus der Definition abzuleiten: $A^{-1} \cdot ( A \cdot \vec v ) = (A^{-1} \cdot A) \cdot \vec v = E \cdot \vec v = \vec v$.

Der Grund für die Beschränkung auf reguläre Matrizen ist oft auf den ersten Blick ersichtlich, wenn man die Auswirkungen der Matrix auf einen Vektor betrachtet. Beispielsweise kann zu der Matrix
\begin{equation}
 \begin{pmatrix}
  1 & 0 & 0 \\
  0 & 0 & 0 \\
  0 & 0 & 1
 \end{pmatrix}
\end{equation}
keine inverse Matrix existieren, denn wenn man einen Vektor mit dieser Matrix transformiert, geht die Information aus der $y$-Koordinate \enquote{verloren}. Somit ist es nachträglich nicht mehr möglich, den ursprünglichen Vektor wiederherzustellen.

In Bezug auf Transposition gilt für die Inversion (ähnlich wie bei der Determinante, siehe Gleichung \ref{transposeddeterminant}):
\begin{equation}
 (A^{-1})^T = (A^T)^{-1}.
\end{equation}

Im Wesentlichen gibt es zwei Verfahren, um die Inverse zu einer Matrix zu berechnen, im Folgenden soll als Ausgangsmatrix
\begin{equation*}
 A = \begin{pmatrix}
	1 & 2 & 0 \\
    2 & 3 & 0 \\
    3 & 4 & 1
 \end{pmatrix}
\end{equation*}
dienen.

Die erste Variante beruht auf dem Gauß-Eliminationsverfahren. Dazu schreibt man die Ausgangsmatrix als Blockmatrix $(A|E)$ mit der Einheitsmatrix an:
\begin{equation}
 \left(\begin{array}{ccc|ccc}
    1 & 2 & 0 &  1 & 0 & 0 \\
    2 & 3 & 0 &  0 & 1 & 0 \\
    3 & 4 & 1 &  0 & 0 & 1
  \end{array}\right)
\end{equation}
Danach wendet man den Gauß-Jordan-Algorithmus auf die Blockmatrix an, bis auf der linken Seite die Einheitsmatrix steht, was in diesem Fall zu folgendem Ergebnis führt:
\begin{equation}
  \left(\begin{array}{ccc|ccc}
    1 & 0 & 0  & -3 & 2 & 0 \\
    0 & 1 & 0  & 2 & -1 & 0 \\
    0 & 0 & 1  & 1 & -2 & 1
  \end{array}\right)
\end{equation}
Auf der rechten Seite kann die inverse Matrix $A^{-1}$ nun direkt abgelesen werden.
% Beispiel von http://de.wikipedia.org/wiki/Regul%C3%A4re_Matrix, korrekt?

Mit diesem Verfahren ist die inverse Matrix recht einfach händisch zu berechnen. Es hat allerdings den Nachteil, dass sich daraus keine \enquote{fertige} Formel ableiten lässt, und es daher für den Einsatz in der Grafikprogrammierung nicht ausreichend schnell zu berechnen ist\footnote{Dies gilt wiederum nur für relativ kleine Matrizen.}.

Daher wird in der 3D-Programmierung meist auf die zweite Variante zurückgegriffen, nach der die Inverse zur Matrix $A$ folgendermaßen definiert ist:
\begin{equation}
 A^{-1} = \frac{1}{\det A} \cdot \adj A
\end{equation}

$\adj A$ bezeichnet hier die \emph{Adjunkte} der Matrix A, die auch als \emph{komplementäre Matrix} bezeichnet wird. Sie besteht aus der transponierten Matrix der Kofaktoren (siehe Gleichung \ref{cofactor}), es gilt also:
\begin{equation}
 \adj A = \begin{pmatrix}
   \tilde a_{11} & \tilde a_{21} & \cdots & \tilde a_{n1}\\
   \tilde a_{12} & \tilde a_{22} & \cdots & \tilde a_{n2}\\
   \vdots & \vdots & \ddots & \vdots\\
   \tilde a_{1n} & \tilde a_{2n} & \cdots & \tilde a_{nn}
 \end{pmatrix}.
\end{equation}

Mit Hilfe dieses Zusammenhangs lassen sich Formeln für Matrizen beliebiger Dimension herleiten. Dabei wird allerdings bald deutlich, dass die Komplexität bzw. die Anzahl der nötigen Rechenoperationen mit der Erhöhung der Dimensionen sehr schnell ansteigt:

Ergibt sich für die Inverse einer $2 \times 2$-Matrix
\begin{equation}
 B = \begin{pmatrix}
  a & b \\
  c & d
 \end{pmatrix}
\end{equation}
noch der relativ übersichtliche Ausdruck
\begin{equation}
 B^{-1} = \frac{1}{ad-bc} \cdot
 \begin{pmatrix}
  d & -b \\
  -c & a
 \end{pmatrix},
\end{equation}
führt das Verfahren für eine $3 \times 3$-Matrix
\begin{equation}
 C = \begin{pmatrix}
  a & b & c \\
  d & e & f \\
  g & h & i
 \end{pmatrix}
\end{equation}
schon zu der wesentlich komplizierteren Formel
\begin{equation}
 C^{-1} = \frac{1}{\det C} \cdot
 \begin{pmatrix}
  ei - hf & ch - bi & bf - ce \\
  fg - di & ai - cg & cd - af \\
  dh - eg & bg - ah & ae - bd
 \end{pmatrix}.
\end{equation}

In Bezug auf die 3D-Grafik am Computer bleibt noch festzustellen, dass die Inversion im Vergleich zu anderen Operationen (\zb zur Matrizenmultiplikation) recht rechenaufwändig ist und daher nicht allzu oft ausgeführt werden sollte (siehe Abschnitt \ref{performance}).

\section{Quaternionen}
Erklärung zu Quaternionen. Nicht ganz alltägliches Konzept.

\subsection{Rechenregeln}
Kurze Zusammenfassung der Rechenregeln für Quaternionen.