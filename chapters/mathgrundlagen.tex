% Kapitel 'Mathematische Grundlagen'
\chapter{Mathematische Grundlagen}
In den folgenden Abschnitten möchte ich die mathematischen Grundprinzipien behandeln, auf denen nahezu alle Konzepte und Operationen der 3D-Grafik aufbauen.

% Kapitel über Vektoren
% Wozu werden welche Vektoren verwendet? Erläuterung homogene Koordinaten. Schreibweisen von Vektoren.
\section{Vektoren}
Vektoren sind wohl die wichtigste Struktur in der 3D-Grafik. Sie werden nicht nur in ihrer dreidimensionalen Form eingesetzt, sondern auch in ihrer zweidimensionalen Form und um eine vierte Koordinate erweitert als Vektoren mit homogenen Koordinaten (dazu später mehr).

Trotz der universellen Verwendung gehen die benötigten Operationen nicht über den Schulstoff hinaus. Insbesondere das Skalarprodukt und das Kreuzprodukt werden sehr häufig benötigt, zum Beispiel um über den Zusammenhang
% Wie 'flacher' setzen?
$\vec{a}\cdot\vec{b} = \left|\vec{a}\right|\left|\vec{b}\right|\cos\alpha$
den von zwei Vektoren eingeschlossenen Winkel zu berechnen oder über das Kreuzprodukt eine Normale eines Polygons zu bestimmen.

Vektoren können dabei auf zwei Arten notiert werden, in Spaltenform
\begin{equation*}
 \vec{v} = \begin{pmatrix} x \\ y \\ z \end{pmatrix},
\end{equation*}
oder in Zeilenform
\begin{equation*}
 \vec{v} = \begin{pmatrix} x & y & z \end{pmatrix}.
\end{equation*}

Scheint der Unterschied zunächst noch rein kosmetischer Natur zu sein, zeigt sich spätestens beim Arbeiten mit Matrizen, dass die Entscheidung für eine der Formen doch erhebliche Konsequenzen nach sich zieht. In der Programmierung werden beide Varianten gleichermaßen verwendet; ich habe mich entschieden, in diesem Dokument Spaltenvektoren zu verwenden, da diese Variante in der Schulmathematik üblich ist.

Eingebettet in Fließtext sind Zeilenvektoren wesentlich platzsparender -- man kann einen Spaltenvektor aber als transponierten Zeilenvektor $\vec{v} = \begin{pmatrix} x & y & z \end{pmatrix}^T$ anschreiben, um diesen Vorteil zu übernehmen (siehe Abschnitt \ref{transposition}).


\subsection{Homogene Koordinaten}
Vierdimensionale Vektoren, also Vektoren mit \emph{homogenen Koordinaten}, dienen hauptsächlich dazu, mithilfe von Matrizen nicht nur lineare, sondern alle affinen Transformationen darstellen zu können.
% Beispiel aus http://de.wikipedia.org/wiki/Homogene_Koordinaten einfügen.
Dazu wird eine vierte Koordinate $w$ eingeführt.
% Hyperebene: Eberly (3D game engine design), S. 9 (2.1.4 Homogeneous Transformations)
% Verwendung von w in Robotik-Matrizen?


\section{Matrizen}
Auf den ersten Blick gleichen Matrizen einfachen Tabellen von Zahlen, es gibt aber einen wesentlichen Unterschied: Für sie sind Rechenoperationen wie Addition und Multiplikation definiert.

Eine Matrix
\begin{equation}
 A = \begin{pmatrix}
   a_{11} & a_{12} & \cdots & a_{1m}\\
   a_{21} & a_{22} & \cdots & a_{2m}\\
   \vdots & \vdots & \ddots & \vdots\\
   a_{n1} & a_{n2} & \cdots & a_{nm}
 \end{pmatrix}
 \in \mathbb R^{m,n}
\end{equation}
besteht aus $m$ Zeilen und $n$ Spalten, es handelt sich um eine $m \times n$-Matrix (sprich: \emph{m kreuz n}).

Matrizen mit $m = n$ werden als \emph{quadratische Matrizen} bezeichnet.
% Besondere Eigenschaften der quadratischen Matrizen? ``Einige Operationen sind nur für quadratische Matrizen definiert, ...''

Als \emph{Hauptdiagonale} einer Matrix wird die Linie vom linken obersten Element schräg zum rechten untersten Element bezeichnet; die Elemente der Hauptdiagonale sind also $a_{11}, a_{22}, \cdots, a_{nn}$. Matrizen, die außerhalb der Hauptdiagonale nur nur Elemente mit dem Wert 0 haben, werden als \emph{Diagonalmatrizen} bezeichnet. Sie lassen einige Vereinfachungen in den Berechnungen zu, auf die aber an dieser Stelle nicht wieter eingegangen werden soll.
% siehe http://de.wikipedia.org/wiki/Diagonalmatrix.

\subsection{Addition}
Zwei Matrizen werden addiert, indem man jeweils die Einträge der beiden Matrizen addiert:
\begin{align}
 (A + B)_{ij} = a_{ij} + b_{ij}%\\
% \nonumber\text{mit }1 \leq i \leq m \text{ und } 1 \leq j \leq n.
\end{align}
% Quelle für Notation: http://en.wikipedia.org/wiki/Matrix_(mathematics)
mit $1 \leq i \leq m$ und $1 \leq j \leq n$.

Folglich ist die Addition zweier Matrizen nur dann definiert, wenn sie beide die gleichen Dimensionen $m \times n$ haben.

Die Matrizenaddition ist assoziativ und kommutiativ. Es ist leicht zu erkennen, dass die Matrizenaddition ein neutrales Element besitzt: analog zur Null bei der Addition von Skalaren eine Matrix deren Elemente alle 0 sind, kurz Nullmatrix genannt.

\subsection{Multiplikation}
Das Produkt $C$ zweier Matrizen $A \in \mathbb R^{m,o}$ und $B \in \mathbb R^{o,n}$ ist folgendermaßen definiert:
\begin{equation}
 c_{ij} = \sum_{k=1}^m{a_{ik} \cdot b_{kj}}
\end{equation}
mit $1 \leq i \leq m$ und $1 \leq j \leq n$.

Etwas anschaulicher formuliert erhält man das $i$-te Element der $j$-ten Spalte des Ergebnisses, indem man das Punktprodukt der $i$-ten Zeile der linken Matrix mit der $j$-Zeile der rechten Matrix bildet -- kurz \enquote{Zeile mal Spalte}.

Es müssen also die Spaltenanzahl der linken Matrix und die Zeilenanzahl der rechten Matrix gleich sein, damit eine Multiplikation möglich ist. Das entstehende Produkt hat dabei die Dimensionen $m \times n$, also die Zeilenanzahl der linken Matrix und die Spaltenanzahl der rechten Matrix.

Ein Beispiel zur Veranschaulichung:
\begin{equation}
\begin{split}
 \begin{pmatrix}
  0 & 1 & 2 \\
  3 & 4 & 5
 \end{pmatrix}
 \cdot
 \begin{pmatrix}
  6 & 7 \\
  8 & -9 \\
  -10 & 11
 \end{pmatrix}
 =
 \begin{pmatrix}
  0 \cdot 6 + 1 \cdot 8 + 2 \cdot (-10) & 0 \cdot 7 + 1 \cdot (-9) + 2 \cdot 11 \\
  3 \cdot 6 + 4 \cdot 8 + 5 \cdot (-10) & 3 \cdot 7 + 4 \cdot (-9) + 5 \cdot 11
 \end{pmatrix}\\
 =
 \begin{pmatrix}
  0 + 8 - 20 & 0 - 9 + 22 \\
  18 + 32 - 50 & 21 - 36 + 55
 \end{pmatrix}
 =
 \begin{pmatrix}
   -12 & 13 \\
   0 & 40
 \end{pmatrix}
\end{split}
\end{equation}

Für die Multiplikation von quadratischen Matrizen gibt es ein neutrales Element, für das
\begin{equation}
 A \cdot E = E \cdot A = A
\end{equation}
gilt. $E$ wird als \emph{Einheitsmatrix} bezeichnet und ist eine Diagonalmatrix, deren Elemente entlang der Hauptdiagonale alle den Wert 1 haben.

Fasst man einen $n$-dimensionalen Spaltenvektor als $n \times 1$-Matrix auf, so kann die Multiplikation einer Matrix mit einem Vektor als Spezialfall der Matrizenmultiplikation gesehen werden. Sie wird auch als \emph{Transformation} des Vektors bezeichnet. Ergebnis der Operation ist wiederum ein $n$-dimensionaler Vektor.


\subsection{Transposition}
\label{transposition}
Erklärung der Transposition einer Matrix.

\subsection{Determinante und Inverse}
Definition der Inversen einer Matrix. Um sie berechnen zu können, nimmt man die Determinante zur Hilfe.

\section{Quaternionen}
Erklärung zu Quaternionen. Nicht ganz alltägliches Konzept.

\subsection{Rechenregeln}
Kurze Zusammenfassung der Rechenregeln für Quaternionen.