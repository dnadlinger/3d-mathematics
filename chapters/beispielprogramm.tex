% Anhang 'Beispielprogramm'
\chapter{Beispielprogramm}

Begleitend zu dieser Arbeit ist das Projekt \enquote{d4} entstanden, ein kleiner Software-Renderer in der Programmiersprache D.

Bei einem Software-Renderer handelt es sich um ein Programm, dass die 3D-Pipeline, die den Programmierern wie oben schon erwähnt heute in hoch spezialisierter und optimierter Form als Silizium-Chip zu Verfügung steht\footnote{Damit ist natürlich die Grafikkarte gemeint!}, ganz in Software, also auf dem Hauptprozessor des Computers, implementiert. Dies kann zwar durchaus auch in Produktiv-Software geschehen, wenn zum Beispiel besonders hohe Anforderungen an die Genauigkeit gestellt werden oder spezielle Berechnungsverfahren zum Einsatz kommen sollen, ist aber spätestens seit Mitte der 90er-Jahre überholt (\vgl \citep{wiki:grafikkarte}).
% Jahreszahlen: http://de.wikipedia.org/wiki/Grafikkarte

Nichtsdestotrotz regt die Aufgabenstellung dazu an, sich eingehender mit den \enquote{Innereien} der 3D-Datenverarbeitung zu beschäftigen, über die man normalerweise dank abstrahierter Programmierschnittstellen nicht weiter nachdenken muss.

Im Verzeichnis \path{soundso} befinden sich die ausführbaren Dateien für Microsoft Windows, Mac OS X und Linux.

Die Quellen des Programms befinden sich unter \path{wasauchimmer}, die aus den Quellen erstellte Dokumentation in \path{woauchimmer}. Mit Hilfe dieser und jener Tools sollte es möglich sein, den Quellcode beispielsweise nach Anpassungen erneut zu übersetzten. Bitte beachten Sie dabei, dass Paket XYZ installiert sein muss.

Die Dokumentation des Quellcodes sollte ausreichend genau sein, um Ihnen zu ermöglichen, im Quellcode zu stöbern. Ich habe bei der Entwicklung besonders auf eine klare Struktur geachtet, auch wenn es manchmal zum Nachteil der Leistung des Systems ist.