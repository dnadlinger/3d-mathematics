% Kapitel 'Einleitung'
\chapter{Einleitung}

Die vorliegende Fachbereichsarbeit beschäftigt sich mit den mathematischen Grundlagen und Hintergründen der \emph{3D-Grafik}. Unter diesem Begriff wird jenes Teilgebiet der \emph{Computergrafik} verstanden, das die Generierung von zweidimensionalen Bildern aus der Beschreibung einer dreidimensionalen Umgebung im weitesten Sinne umfasst. Die Einsatzgebiete der 3D-Grafik sind während der letzten Jahre mit der steigenden Verfügbarkeit von leistungsfähigen Computersystem stetig gewachsen. Heute gehören dazu natürlich Computerspiele und Spezialeffekte in Filmen, aber auch die Visualisierung von Messdaten in der Medizin, die Darstellung von dreidimensionalen Plänen im CAD\footnote{von engl. \emph{Computer Aided Design}, \enquote{Computerunterstützter Entwurf}. Erstellung von Modellen und Plänen mit Hilfe von Computer-Software, etwa im Maschinenbau.}-Bereich und vieles mehr.

In vielen Verfahren und Algorithmen der 3D-Grafik kommen zwei mathematische Konzepte zur Anwendung, die in der Schule nicht behandelt werden: Matrizen und Quaternionen. Ich werde deshalb in Kapitel \ref{mathgrundlagen} auf die im weiteren Verlauf der Arbeit angewendeten Definitionen und Sätze eingehen. In diesem Zusammenhang möchte ich gleich darauf hinweisen, dass der Fokus dieser Arbeit ausdrücklich nicht auf den zahlreichen Eigenschaften dieser Strukturen liegt, sondern auf deren Anwendung in der 3D-Grafik. Diese können samt den Beweisen zu den verschiedenen Sätzen und Rechenregeln in jedem Standardwerk zur Linearen Algebra nachgeschlagen werden, ich werde daher an einigen Stellen auf deren Wiedergabe verzichsten. Allein mit den Herleitungen aller in den verschiedenen Teilbereichen der Mathematik wichtigen Sätze zur Matrizenrechnung ließen sich sonst wohl ganze Buchbände füllen.

% Beschränkung auf Echtzeit-Grafik

In Kapitel \ref{grafikgrundlagen} werde ich darauf versuchen, einen kurzen Überblick über die Umsetzung der 3D-Grafik am Computer zu geben. Dazu zählen natürlich in erster Linie die verschiedenen theoretischen Herangehensweisen, aber auch die praktische Realisierung in Form von Hardware wird kurz zur Sprache kommen.

Die beiden darauffolgenden Kapitel haben die verschiedenen Transformationen zum Thema, welche die Objekte auf dem Weg zum fertigen Bild durchlaufen. Kapitel \ref{3d-transformations} behandelt dabei die Skalierung, die Translation und die verschienenen Möglichkeiten, um Rotationen darzustellen -- kurz gesagt also die Operationen, um die Platzierung und Ausrichtung eines Objekts in der virtuellen Umgebung festzulegen. In Kapitel \ref{viewingtransformations} werde ich genauer auf die Operationen eingehen, mit deren Hilfe schließlich eine Ansicht der virtuellen Welt auf den Bildschirm gebracht wird.

Im Rahmen dieser Fachbereichsarbeit habe ich mich nicht nur theoretisch mit dem Themenkomplex 3D-Grafik auseinandergesetzt. Begleitend zu der Arbeit am Text ist auch ein Projekt namens \emph{d4} entstanden, das die praktische Umsetzung der mathematischen Grundlagen zeigt. In Anhang \ref{beispielprogramm} werde ich Aufbau und Funktionsweise dieses Programms kurz beschreiben.

% Es muss immer vereinfacht werden.
% Mathematik ist \enquote{schwarze Magie} in der Spieleprogrammierung.