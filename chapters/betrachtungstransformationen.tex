% Kapitel 'Betrachtungstransformationen'
\chapter{Betrachtungstransformationen}
\label{viewtransformation}
Im vorigen Kapitel haben wir die verschiedenen Formen von Skalierung, Translation und Rotation behandelt, die für die Plazierung und Ausrichtung der einzelnen Modelle in der Szene, also für das Berechnen der World Matrix, nötig sind. Das Thema dieses Kapitel ist die Transformationen, die für den Übergang vom Weltkoordinatensystem in das Bildschirmkoordinatensystem verwendet werden.

\section{View Matrix}
\label{view}
In der 3D-Programmierung haben sich zwei praktische Arten eingebürgert, um die Ansicht auf die Szene festzulegen: Die Position des Betrachters kombiniert mit seiner Blickrichtung oder einem Punkt, den er anvisiert. Die View Matrix, die aus diesen Parametern berechnet wird, überführt das Weltkoordinatensystem in das Kamerakoordinatensystem, das seinen Ursprung im Auge des virtuellen Betrachters hat. Die $x$-Achse des neuen Systems zeigt dabei nach rechts, die $y$-Achse nach oben, und die $z$-Achse (in einem rechtshändigen System) gegen die Blickrichtung.

Im ersten Fall sind drei Vektoren gegeben: der Positionsvektor $\vec p$, der Richtungsvektor $\vec d$, und der Hochvektor $\vec u$. Die Bedeutung von $\vec p$ und $\vec u$ sollte klar sein, der Hochvektor ist nötig, um festzulegen, welche Richtung für den Betrachter \enquote{oben} ist. Alle Vektoren haben Einheitslänge, $\vec u$ steht senkrecht auf $\vec d$. Der dritte Basisvektor des neuen Koordinatensystems, der Rechtsvektor $\vec r$, lässt sich leicht über das Kreuzprodukt berechnen:
\begin{equation}
 \vec r = \vec d \times \vec u
\end{equation} 

Im zweiten Fall ist neben $\vec p$ nur die Position des Ziels, $\vec t$, gegeben. Der Hochvektor muss aus dem Hochvektor $\vec u_w$ des Weltkoordinatensystems (normalerweise $\begin{pmatrix} 1 & 0 & 0 \end{pmatrix}^T$) berechnet werden. In der Regel wird diese Darstellung in die erste Variante konvertiert, um dann daraus die Matrix zu berechnen. Dazu wird zunächst der normierte Richtungsvektor $\vec d$ berechnet, der von der Position der Kamera zum Ziel zeigt:
\begin{equation}
 \vec d = \frac{\vec t - \vec p}{\lvert \vec t - \vec p \rvert}
\end{equation}
Der Rechtsvektor muss senkrecht auf der zwischen Welt-Hochvektor und Richtungsvektor aufgespannten Ebene stehen, lässt sich über das Kreuzprodukt gewinnen:
\begin{equation}
 \vec r = \frac{\vec d \times \vec u_w}{\lvert \vec d \times \vec u_w \rvert}
\end{equation} 
Der Hochvektor berechnet sich nun als 
\begin{equation}
 \vec u = \vec r \times \vec d = \frac{\vec d \times \vec u_w \times \vec d}{\lvert \vec d \times \vec u_w \times \vec d \rvert},
\end{equation} 
was sich unter Zuhilfenahme der Grassmann-Identität\footnote{$\vec a \times ( \vec b \times \vec c ) = \vec b \cdot ( \vec a \cdot \vec c ) - \vec c \cdot ( \vec a \cdot \vec b)$} zu dem schneller zu berechnenden Ausdruck
\begin{equation}
 \vec u = \frac{\vec u_w - ( ( \vec u_w \cdot \vec d ) \cdot \vec d )}{\lvert \vec u_w - ( ( \vec u_w \cdot \vec d ) \cdot \vec d ) \rvert}
\end{equation} 
vereinfachen lässt.

In beiden Fällen geht es nun darum, aus $\vec p$, $\vec d$, $\vec u$ und $\vec r$ die Transformationsmatrix zu bestimmen. Die Transformation in das Kamerakoordinatensystem kann in zwei Schritte aufgeteilt werden: Zuerst wird der Ursprung des Koordinatensystems in den Standpunkt der Kamera verschoben, dann werden die Achsen korrekt ausgerichtet.

Um den Ursprung in $\vec p$ zu verschieben, müssen alle Objekte um $-\vec p$ verschoben werden:
\begin{equation}
 M_{trans} = \begin{pmatrix}
  1 & 0 & 0 & -p_x \\
  0 & 1 & 0 & -p_y \\
  0 & 0 & 1 & -p_z \\
  0 & 0 & 0 & 1
 \end{pmatrix}
\end{equation}

Die Vektoren $-\vec d$, $\vec u$ und $\vec r$ sind ja die Basisvektoren des Kamarakoordinatensystems, die Matrix für den Rotationsteil kann also wie in Kapitel \ref{rotationbasevectors} beschrieben aufgestellt werden:
\begin{equation}
 M_{rot} =
 \begin{pmatrix}
  r_x & u_x & -d_x & 0 \\
  r_y & u_y & -d_y & 0 \\
  r_z & u_z & -d_z & 0 \\
  0 & 0 & 0 & 1
 \end{pmatrix}
\end{equation}

Um die fertige View Matrix zu erhalten, müssen die beiden Matrizen nur noch multipliziert werden:
\begin{equation}
\begin{split}
 M_{view} &= M_{rot} \cdot M_{trans} \\
 &=
 \begin{pmatrix}
  1 & 0 & 0 & -p_x \\
  0 & 1 & 0 & -p_y \\
  0 & 0 & 1 & -p_z \\
  0 & 0 & 0 & 1
 \end{pmatrix} \cdot
 \begin{pmatrix}
  r_x & u_x & -d_x & 0 \\
  r_y & u_y & -d_y & 0 \\
  r_z & u_z & -d_z & 0 \\
  0 & 0 & 0 & 1
 \end{pmatrix} \\
 &=
 \begin{pmatrix}
  r_x & u_x & -d_x & -\vec r \cdot \vec p \\
  r_y & u_y & -d_y & -\vec u \cdot \vec p \\
  r_z & u_z & -d_z & \vec d \cdot \vec p \\
  0 & 0 & 0 & 1
 \end{pmatrix}
\end{split}
\end{equation}

Culling

\section{Projection Matrix}
\label{projection}
Clipping

Viewport-Koordinaten