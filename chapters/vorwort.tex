% Vorspann 'Vorwort'
\chapter*{Vorwort}

Heutzutage begegnen uns ständig computergenerierte Bilder, sei es in Form von Spezialeffekten oder künstlichen Charakteren in einem Kinofilm, Computerspielen oder der Visualisierung von Bauplänen, lange bevor die ersten Bagger rollen. Diese Möglichkeit, virtuelle Welten im Auge eines Betrachters Realität werden zu lassen, macht die 3D-Grafik in meinen Augen zu einem der interessantesten Anwendungsgebiete der Mathematik.

Als ich mich vor mittlerweile über fünf Jahren zum ersten Mal mit dieser Thematik beschäftigte, faszinierte mich vor allem die Entdeckung, dass hinter all den fantastischen Bildern letzendlich doch nichts anderes als \enquote{trockene} Mathematik steckt. Bei meinen ersten Versuchen, diese Mechanismen zu verstehen, war mir neben einigen guten Büchern vor allem mein Vater behilflich, der sich redlich bemühte, mir das nötige mathematische Handwerkszeug beizubringen, als selbst die trigonometrischen Funktionen noch Neuland für mich waren. An dieser Stelle ein herzliches Dankeschön dafür!

Während der letzten Jahre setzte ich mich immer wieder mit dem Thema auseinander, so dass ich heute mit Fug und Recht behaupten kann, die Vorgänge hinter den Kulissen von Computerspielen oder Modellierungsprogrammen zu begreifen. Ein Bereich blieb aber lange Zeit eine Art \enquote{blinder Fleck}: die genaue Bedeutung der zahlreichen mathematischen Hilfsmitel. Obwohl ich etwa zahlreiche Transformationen in meinen kleinen Projekten anwendete, hatte ich sie aber immer als gegeben betrachtet -- die Herleitungen auch tatsächlich zu verstehen, schien mir ein hoffnungsloses Unterfangen zu sein. Mit der Zeit aber zeichneten sich für mich viele mathematische Zusammenhänge immer klarer ab, und bald entstand die Idee, mich endlich einmal intensiver mit den mathematischen Grundlagen der 3D-Grafik auseinanderzusetzen. Mangels konkreter Motivation legte ich diesen Plan aber erst einmal ad acta.

Als ich schließlich gegen Mitte der siebten Klasse von der interessanten Möglichkeit erfuhr, eine Fachbereichsarbeit zu verfassen, dauerte es bei der Suche nach einem geeigneten Thema nicht lange, bis ich mich wieder an dieses Vorhaben erinnerte. Hatte ich anfangs noch Zweifel, ob das Thema genug Stoff für eine ganze Arbeit bieten würde, musste ich nach dem Erstellen der ersten Konzepte die Themenstellung sogar explizit auf die Grundlagen eingrenzen musste, da der Umfang der Arbeit auszuufern drohte. Vor einigen Monaten begann ich schließlich mit der konkreten Arbeit am theoretischen und praktischen Teil dieses Projektes. Das Ergebnis dieser Bemühungen halten Sie, verehrter Leser, nun in Form dieser Arbeit und der beiligenden CD-ROM in Händen.

Zum Schluss dieses Vorwortes möchte ich mich noch besonders bei meinem Mathematikprofessor Mag.~Herbert Lenz bedanken, der sich in seinem voraussichtlich vorletzten Dienstjahr noch in das Abenteuer gestürzt hat, eine Fachbereichsarbeit zu betreuen. Obwohl die Korrekturarbeiten sicher mehr als nur ein paar Abende seiner wohlverdienten Freizeit beansprucht haben dürften, bekam ich stets gute Ratschläge und Verbesserungsvorschläge zu hören. Danke!

\vspace{0.5cm}
Linz, im Februar 2009

\vspace{0.5cm}
{\small(David Nadlinger)}