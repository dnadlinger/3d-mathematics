% Vorspann 'Vorwort'
\chapter{Vorwort}

Die 3D-Grafik ist ein Gebiet, für das ich mich schon lange interessiere -- eröffnet es doch die Möglichkeit, virtuelle Welten im Auge eines Betrachters Realität werden zu lassen. Gerade in unserer heutigen Zeit begegnen uns ständig computergenierierte Bilder, sei es in Form von Spezialeffekten oder künstlichen Chakateren in einem Kinofilm, einem Computerspiel oder der Visualisierung von Architektur vor dem Bau.

Als ich mich vor mittlerweile über fünf Jahren zum ersten Mal mit der Thematik beschäftigte, faszinierte mich die Entdeckung, dass hinter den fantastischen Bildern letzendlich doch nichts anderes als \enquote{trockene} Mathematik steckt. Bei den folgenden Suche nach den Zusammenhängen war mir neben einigen guten Büchern vor allem mein Vater behilflich, der sich redlich bemüht hat, mir das nötige mathematische Handwerkszeug beizubringen, als selbst die trigonometrischen Funktionen noch Neuland für mich waren. An dieser Stelle ein herzliches Dankeschön dafür!

Während der letzten Jahre setzte ich mich immer wieder mit dem Thema auseinander, so dass ich mit Fug und Recht behaupten konnte, die Vorgänge hinter den Kulissen von Computerspielen oder Modellierungsprogrammen zu verstehen. Obwohl ich sie munter anwendete, hatte ich viele der mathematischen Hilfsmittel wie bestimmte Transformationen aber immer als gegeben betrachtet und die Herleitungen nicht weiter zu verstehen versucht, weil sie mir zu kompliziert erschienen. Mit der Zeit zeichneten sich für mich viele mathematische Zusammenhänge immer klarer ab, und bald endstand die Idee, mich endlich einmal intensiver mit den mathematischen Grundlagen der 3D-Grafik auseinanderzusetzen. Mangels konkreter Motivation legte ich diesen Plan aber erst einmal ad acta.

Als ich schließlich gegen Mitte der siebten Klasse von der interessanten Möglichkeit erfuhr, eine Fachbereichsarbeit als Vorprüfung für die Matura zu verfassen, fiel mir dieses Vorhaben bei der Suche nach einem geeigneten Thema schnell wieder ein. Anfangs bestanden noch Zweifel, ob das Stoffgebiet groß genug sein würde, um eine ganze mathematische Arbeit zu füllen. Bei der genaueren Planung des Aufbaus stellte sich aber heraus, dass in das Themengebiet noch viele andere interessante Bereiche fallen würden. Schließlich musste ich mich in der Themenstellung sogar explizit auf die Grundlagen beschränken, da die Länge der Arbeit auszuufern drohte.

Zu guter Letzt möchte ich noch meinem Mathemathikprofessor Mag. Herbert Lenz danken, der sich in seinem voraussichtlich vorletzten Dienstjahr noch in das Abenteuer gestürzt hat, eine Fachbereichsarbeit zu betreuen. Obwohl die Korrekturarbeiten sicher so manche Abende seiner wohlverdienten Freizeit beansprucht haben dürften, bekam ich stets gute Ratschläge und Verbesserungsvorschläge zu höhren. Danke!

\vspace{1.2cm}
Linz, im Februar 2009

\vspace{0.6cm}
{\small(David Nadlinger)}