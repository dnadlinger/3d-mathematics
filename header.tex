%%%%%%%%%%%%%%%%%%%%%%%%%%%%%%%%%%%%%%%%%%%%%%%%%%%%%%%%%%%%%%%%%%%%%%%
%% Optionen zum Layout des Buchs                                     %%
%%%%%%%%%%%%%%%%%%%%%%%%%%%%%%%%%%%%%%%%%%%%%%%%%%%%%%%%%%%%%%%%%%%%%%%
\documentclass[
%draft,				% überlangen Zeilen in Ausgabe gekennzeichnet
appendixprefix,			% Anhang wird "Anhang" vor die Überschrift gesetzt
parskip=half,
DIV=12,
BCOR=5mm,
%12pt,				% Schriftgröße (12pt, 11pt (Standard))
%BCOR1cm,			% Bindekorrektur, bspw. 1 cm
%DIVcalc,			% führt die Satzspiegelberechnung neu aus (scrguide 2.4)
%oneside,			% einseitiges Layout
%openany,			% Kapitel können auch auf linken Seiten beginnen
%headsepline,			% Trennline zum Seitenkopf
%footsepline,			% Trennline zum Seitenfuß
%notitlepage,			% in-page-Titel, keine eigene Titelseite
%chapterprefix,			% vor Kapitelüberschrift wird "Kapitel Nummer" gesetzt
%normalheadings,		% Überschriften etwas kleiner (smallheadings)
%idxtotoc,			% Index im Inhaltsverzeichnis
%liststotoc,			% Abb.- und Tab.verzeichnis im Inhalt
%bibtotoc,			% Literaturverzeichnis im Inhalt
%leqno,				% Nummerierung von Gleichungen links
%fleqn				% Ausgabe von Gleichungen linksbündig
]
{scrbook}

\usepackage{fancyhdr}
%\pagestyle{fancy}

%% Deutsche Anpassungen %%%%%%%%%%%%%%%%%%%%%%%%%%%%%%%%%%%%%
\usepackage[ngerman]{babel}
\usepackage[utf8]{inputenc}
\usepackage[babel,german=guillemets]{csquotes}
\usepackage[T1]{fontenc}

\usepackage[square]{natbib}

\usepackage{amsmath, amsthm, amssymb}


\usepackage{graphicx} %%Zum Laden von Grafiken: .jpg .png .pdf .mps
%\usepackage{subfig} %%Teilabbildungen in einer Abbildung
%\usepackage{pst-all} %%PSTricks - nicht verwendbar mit pdfLaTeX

% PDF-Inhaltsverzeichnis und Links
\usepackage[colorlinks,linkcolor=blue]{hyperref}

%\newcommand{\path}[1]{\texttt{#1}}
\newcommand{\adj}[1]{\operatorname{adj}#1}
\newcommand{\zb}{z.\thinspace B.\ }
\newcommand{\vgl}{vgl.\thinspace }