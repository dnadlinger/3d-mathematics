\documentclass[
DIV=13,
BCOR=8mm,
11pt,
parskip=half,
%appendixprefix,
%draft,
]
{scrbook}

%\usepackage{fancyhdr}
%\pagestyle{fancy}


% German language settings
\usepackage[ngerman]{babel}
\usepackage[utf8]{inputenc}
\usepackage[babel,german=guillemets]{csquotes}


% Bibliography and citations
\renewcaptionname{ngerman}{\bibname}{Verwendete Literatur}
\newcommand{\vgl}{vgl. \thinspace{}}
\usepackage[]{natbib}
\bibliographystyle{dinat_custom}
\newcommand{\ccite}[1]{\citet{#1}}
\newcommand{\vglr}[1]{\citep[vgl.][]{#1}}
\newcommand{\vgls}[2]{\citep[vgl.][#2]{#1}}
\newcommand{\vglrt}[1]{\citealp[vgl.][]{#1}}
\newcommand{\vglst}[2]{\citealp[vgl.][#2]{#1}}
\newcommand{\vglsd}[4]{(\vgl \citealp[#2]{#1}; \citealp[#4]{#3})}
%\usepackage{bibgerm}
%\usepackage[square, numbers]{natbib}
%\bibliographystyle{geralpha}
%\newcommand{\ccite}[1]{\citep{#1}}
%\newcommand{\vglr}[1]{(\vgl \citep{#1})}
%\newcommand{\vgls}[2]{(\vgl \citep{#1}, #2)}
%\newcommand{\vglrt}[1]{\citep{#1}}
%\newcommand{\vglst}[2]{\citep{#1}, #2}
%\newcommand{\vglsd}[4]{(\vgl \citep{#1}, #2; \citep{#3}, #4)}


% Font settings
\usepackage[T1]{fontenc}
\usepackage{textcomp}
\usepackage{microtype}
\usepackage[fullfamily,opticals,footnotefigures,mathlf,mathtabular]{MinionPro}
\renewcommand{\sfdefault}{\rmdefault}
\usepackage[toc,bib,eqno]{tabfigures}
% minionpro replaces these packages:
%\usepackage{amsmath, amssymb}


% Captions, footnotes, ...
\usepackage[center,footnotesize,it]{caption}
%\renewcaptionname{ngerman}{\figurename}{Abb.}
%\renewcommand{\thefootnote}{\fnsymbol{footnote}}


% Graphics
\usepackage{graphicx}
\usepackage{wrapfig}
%\usepackage{subfig}
\graphicspath{{./images/}}


% TOC and links in PDF
\usepackage[pdftex,unicode]{hyperref}
\hypersetup{
    unicode=true,          % non-Latin characters in Acrobat’s bookmarks
    pdftoolbar=true,        % show Acrobat’s toolbar?
    pdfmenubar=true,        % show Acrobat’s menu?
    pdffitwindow=false,     % window fit to page when opened
    pdfstartview={FitH},    % fits the width of the page to the window
    pdftitle={Mathematische Grundlagen der 3D-Grafik},    % title
    pdfauthor={David Nadlinger},     % author
    pdfsubject={Mathematik/3D-Grafik},   % subject of the document
%    pdfcreator={Creator},   % creator of the document
%    pdfproducer={Producer}, % producer of the document
%    pdfkeywords={keywords}, % list of keywords
    pdfnewwindow=true,      % links in new window
    colorlinks=true,       % false: boxed links; true: colored links
    linkcolor=black,          % color of internal links
    citecolor=black,        % color of links to bibliography
    filecolor=black,      % color of file links
    urlcolor=black           % color of external links
}


% Custom commands
%\newcommand{\path}[1]{\texttt{#1}}
\newcommand{\argument}[1]{\texttt{#1}}
\newcommand{\key}[1]{\texttt{#1}}
\newcommand{\adj}[1]{\operatorname{adj}#1}
\newcommand{\zb}{z.\thinspace B.\ }
\newcommand{\textvec}[3]{{\footnotesize$\begin{pmatrix} #1 & #2 & #3 \end{pmatrix}^T$}}
\newcommand{\textvech}[4]{{\footnotesize$\begin{pmatrix} #1 & #2 & #3 & #4 \end{pmatrix}^T$}}
\newcommand{\missing}{\ldots}